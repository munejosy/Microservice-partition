\documentclass[conference]{IEEEtran}
\usepackage{gensymb}
\usepackage{graphicx}
\usepackage{colortbl}
\usepackage{todonotes}
% Macros for proof-reading
\usepackage[normalem]{ulem} % for \sout
\usepackage{xcolor}
\newcommand{\ra}{$\rightarrow$}
\newcommand{\ugh}[1]{\textcolor{red}{\uwave{#1}}} % please rephrase
\newcommand{\ins}[1]{\textcolor{blue}{\uline{#1}}} % please insert
\newcommand{\del}[1]{\textcolor{red}{\sout{#1}}} % please delete
\newcommand{\chg}[2]{\textcolor{red}{\sout{#1}}{\ra}\textcolor{blue}{\uline{#2}}} % please change

% Put edit comments in a really ugly standout display
\usepackage{ifthen}
\usepackage{amssymb}
\newboolean{showcomments}
\setboolean{showcomments}{true} % toggle to show or hide comments
\ifthenelse{\boolean{showcomments}}
  {\newcommand{\nb}[2]{
    \fcolorbox{gray}{yellow}{\bfseries\sffamily\scriptsize#1}
    {\sf\small$\blacktriangleright$\textit{#2}$\blacktriangleleft$}
   }
   \newcommand{\version}{\emph{\scriptsize$-$working$-$}}
  }
  {\newcommand{\nb}[2]{}
   \newcommand{\version}{}
  }

\newcommand\joselyn[1]{\nb{Joselyn}{#1}}
\newcommand\doreen[1]{\nb{Doreen}{#1}}
\newcommand\benjamin[1]{\nb{Benjamin}{#1}}

%-------------------------------------------
\newcommand{\todoBK}[1]{\todo[color=green!40]{\small #1 (BK)}}
\newcommand{\todoBKinline}[1]{\todo[color=green!40,inline]{\small #1 (BK)}}
\newcommand{\todoDT}[1]{\todo[color=blue!40]{\small #1 (DT)}}
\newcommand{\todoJM}[1]{\todo[color=yellow!40]{\small #1 (JM)}}
  
%\title{Partitioning Microservices: A domain Engineering Approach}
\title{Challenges in partitioning Microservices : A domain Engineering Approach}


%\author{ Doreen Tuheirwe-Mukasa \\ 			\IEEEauthorblockN{Department of Informatics \\University of Bergen \\ dtu003@student.uib.no}
  %  }
    
\begin{document}
\maketitle
\todoBKinline{We have started to use a certain set of Macros to collaboratively work on Latex files. 
Here is how to use them.
Write backslash-your-name to add a comment: \doreen{Doreen's comment}. You can also suggest edits (but feel free to change stuff 
directly - sometimes it is good to mark something for later discussion. Use ins, del, chg commands as follows: \ins{new text}, 
\del{old text}, \chg{old text}{new text}. Use ugh to highlight stuff for revisiting \ugh{needs improvement}.}


\benjamin{We could even think of splitting the paper into two  (1) One concentrating on challenges of splitting micro services (2) second we apply a domain engineering approach to resolve the issues}
\begin{abstract}
\noindent
In this paper, we attempt to define a micro service and show how micro services can be partitioned based on a domain driven engineering approach.

\end{abstract}

\IEEEpeerreviewmaketitle

\section{Introduction} \label{Introduction}
Micro-services is the new buzz word around software architecture patterns today. Micro-services provide several advantages over monolithic systems. Some of these include the ability to make rapid functional changes which contributes to achieving high integrity factors such as maintainability and scalability, continuous software delivery, and delivering software into production \cite{thones2015microservices}. 

Micro-service introduce a new application design strategy that uses independent fine grained services to compose an application, compared to monolithic application style. In monolithic an application is a single large repository of codebase \cite{Hasselbring2016}. The challenge with that is the difficulty to decompose a system for partial upgrade \cite{Le2014} Micro-service is mainly used in cloud to deploy large and medium applications as a set of small independent services that can be developed, tested, deployed, scaled, operated and upgraded independently. Micro-service in cloud are partitioned so that the services must have ways to register themselves, be discovered by other services, record their configuration, and be generally orchestrated in their deployment and update processes\cite{Villamizar2016}  

 
\noindent
However, a major question and challenge is on how to introduce micro-services, and arrive at an appropriate size for a micro-service \cite{namiot2014micro,fowler2014microservices}. A major question to be answered is where component boundaries should lie on \cite{fowler2014microservices}.

\noindent
Some suggestions have been proposed to this effect, including among others, aspects on if the micro-service will be a user service, and therefore a decision being made based on the tooling (with leaning towards the usage of lightweight tools), size being determined by the number of lines of code (with recommendations of not exceeding a couple thousand lines of code), and functionality in terms of the micro-service accomplishing specifically only one task \cite{thones2015microservices}. \cite{namiot2014micro} proposes partitioning services by use case. Other strategies are to partition by verbs, nouns, or resources, and the scaling cube \cite{abbott2009art}. According to \cite{newman2015building}, independent services should focus service boundaries on	business	boundaries, so as to avoid the difficulties introduced when the service becomes too large. He also postulates that a micro service as something	that	could	be rewritten	in	two	weeks, with proper alignment to team structures.

Size for a microservice is important, because the smaller the service, the more the benefits of micro-service architecture are maximized \cite{newman2015building}. However, there is a lot of ambiguity around the right size  of a micro-service, and there is lack of good guidelines for  designing a micro-service in  terms of scope or size. The challenges that arise encompass how a micro-service can be partitioned in the right size to ensure loose coupling, such that the service can easily be changed to keep up with business and technical demands.

\noindent
According to \cite{owen2016three}, three keys to successful microservices are componentization, collaboration, and reliable connections and controls.

\noindent
That not withstanding, the microservice approach must contend with some issues such as integration between communicating applications \cite{thones2015microservices}, and complexities that arise from creating a distributed system. These include testing, deployment and increased memory consumption \cite{namiot2014micro}. \cite{fowler2014microservices} cites the need for microservices to design for failure by possibly, automatically restoring the failed service.

Domain Driven Design (DDD) provides a number of useful patterns for dealing with the kind of complexity encountered in designing distributed systems and with large and complex domains, by breaking the domain into a series of bounded contexts.

\section{Related Work} \label{Related Work}
\textbf{Something on Microservices building blocks (Joselyne)}

A micro-service is small and focused on doing one thing. Decomposing an application to a micro-service is a crucial task. the following is different strategy to sizing a microservice.

\textbf{Line of code} Some developers relate the size of a micro-service to the number of lines of code (LOC). They recommend that a service should not exceed(10 to 100 LOC) \cite{schermann2015all}. However, this practice is not genuine because micro-services are built using different technology stacks, which differ on LOC, secondary service are deffer to each other on minimum LOC depending on the type of service. For example process service that coordinate call between multiple tasks can have 100 to thousand LOC, whether a data service can be implemented by 10 to approximately 100 LOC . 

\textbf{Deployment unit} Micro-service is defined as one kind of development and deployment unit. this service is mainly in cloud as IaaS, PaaS and Security as service to deploy large application as a set of small services that can be deployed , tested, scaled, operated and updated independently. This service is partitioned in the way it can register themselves, be discovered by other services and be generally orchestrated in their deployment and update process\cite{Villamizar2016}.

\textbf{Business capability}: Business capability defines what a system does in enabling the organization's capacity to successfully perform a unique business activity \cite{ulrich2016business}. In agile development, it is an important way of combining data that have something in common such as functionality, rather than using collections of data entities that expose CRUD-style methods. Because it helps to understand demand impact on application architecture at an early stage. In literature, micro-services are built based on the need to address one business capability, or one business functionality at a time. However developers still have a problem on using business capability as boundary of micro-service, on defining which level of functionality a micro-service should fit, so that it can't be too small a service that depends on other services, leading to decrease in its autonomy or too big to lose the dependency.



\textbf{Something on Domain Engineering building blocks (Doreen))}
What in domain engineering helps to breakdown into services, can be used to breakdown?


\section{Procedure} \label{Procedure}
Our approach is concerned with partitioning a micro-service following domain driven 
engineering principles by considering asynchronous and synchronous communication.



%\section{Discussion}

%\section{Conclusion and Future Work} \label{Conclusions}

\bibliographystyle{IEEEtran}
\bibliography{Partition}

\end{document}